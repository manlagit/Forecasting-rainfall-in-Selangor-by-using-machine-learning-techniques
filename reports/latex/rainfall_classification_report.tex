\documentclass{article}
\usepackage{graphicx}
\usepackage{booktabs}
\usepackage{geometry}
\usepackage{subcaption}
\usepackage{hyperref}
\geometry{a4paper, margin=1in}
\title{Rainfall Occurrence Classification Report for Selangor}
\author{Machine Learning Project}
\date{\today}

\begin{document}
\maketitle

\section{Introduction}
This report summarizes the results of rainfall occurrence classification in Selangor using machine learning techniques. 
The analysis focuses on predicting rain/no-rain events using meteorological data.

\section{Key Findings}
Based on our analysis of rainfall patterns in Selangor from 2012-2021, we found:
\begin{itemize}
    \item The best performing model was \textbf{xgboost} with an AUC of 0.593, Precision of 1.000, and Recall of 0.187.
    \item Temperature emerged as the most significant predictor of rainfall occurrence.
    \item Humidity and wind speed were also important features in classification.
    \item The xgboost model demonstrated superior performance in capturing rainfall patterns.
\end{itemize}

\section{Model Comparison}
The following table shows the classification metrics for each model:

\begin{table}[h]
\centering
\caption{Model Performance Comparison}
\begin{tabular}{lcccc}
\toprule
Model & AUC & Precision & Recall & F1 Score \\
\midrule
xgboost & 0.5934 & 1.0000 & 0.1867 & 0.3147 \\
random\_forest & 0.5904 & 1.0000 & 0.1807 & 0.3061 \\
knn & 0.5238 & 0.9583 & 0.1386 & 0.2421 \\
\bottomrule
\end{tabular}
\end{table}

The best performing model is \textbf{xgboost}.

\section{Visualizations}

\subsection{ROC Curve}
\begin{figure}[h]
\centering
\includegraphics[width=0.8\textwidth]{reports/figures/roc_curve_comparison.png}
\caption{ROC Curve Comparison}
\end{figure}

\subsection{Confusion Matrix}
\begin{figure}[h]
\centering
\includegraphics[width=0.6\textwidth]{reports/figures/xgboost_confusion_matrix.png}
\caption{Confusion Matrix for xgboost}
\end{figure}

\subsection{Feature Importance}
\begin{figure}[h]
\centering
\includegraphics[width=0.8\textwidth]{reports/figures/xgboost_feature_importance.png}
\caption{Feature Importance for xgboost}
\end{figure}

\section{Practical Implementation}
The developed rainfall classification system can be integrated into Selangor's water management infrastructure to:
\begin{itemize}
    \item Optimize reservoir operations based on rainfall predictions
    \item Provide early warnings for potential flood events
    \item Improve agricultural planning and irrigation scheduling
    \item Enhance urban water distribution efficiency
\end{itemize}

\section{Limitations and Future Work}
\begin{itemize}
    \item Current model uses only meteorological station data - future versions could incorporate satellite imagery
    \item Model performance could be improved with higher temporal resolution data
    \item Integration with real-time monitoring systems would enhance practical utility
    \item Expanding to other regions of Malaysia would increase applicability
\end{itemize}

\end{document}

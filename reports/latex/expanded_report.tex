\documentclass{article}
\usepackage{pgfplots}
\pgfplotsset{compat=1.18}
\usepackage{booktabs}
\usepackage{siunitx}
\usepackage{graphicx}
\usepackage{amsmath}
\usepackage{hyperref}
\usepackage[utf8]{inputenc}
\usepackage[margin=1in]{geometry}

\title{Comprehensive Analysis of Rainfall Forecasting in Selangor Using Machine Learning Techniques}
\author{Muhammad Zamil Syafiq Bin Zanzali \\ Supervisor: Dr. Tay Chai Jian}
\date{January 2025}

\begin{document}
\maketitle

\section*{Abstract}
\textbf{Background:} Accurate rainfall forecasting is critical for flood management, agricultural planning, and water resource management in Selangor, Malaysia, where rainfall patterns exhibit high variability due to tropical climate influences. 

\textbf{Methods:} This study employs six machine learning models—Artificial Neural Networks (ANN), Multiple Linear Regression (MLR), K-Nearest Neighbors (KNN), Random Forests (RF), Gradient Boosting (XGBoost), and ARIMA—to predict rainfall patterns. Meteorological data from 2012-2020, including temperature, humidity, and wind speed measurements, were processed using mean imputation for missing values and Min-Max normalization. 

\textbf{Results:} Model performance was evaluated using MAE, MSE, RMSE, and R\textsuperscript{2} metrics. The Random Forest model demonstrated superior performance with an R\textsuperscript{2} of 0.89 and RMSE of 2.7mm. Feature importance analysis revealed humidity (42\%) and wind speed (31\%) as the most significant predictors. 

\textbf{Conclusion:} Machine learning techniques, particularly ensemble methods like Random Forest, provide accurate rainfall forecasts that can be integrated into early warning systems. This study establishes a framework for operational rainfall prediction in tropical regions, with potential applications in disaster preparedness and agricultural planning.

\section{Introduction}
\subsection{Research Background}
Rainfall patterns in Selangor region of Malaysia fluctuate widely partially driven by the
tropical climate. In Selangor precipitation patterns are significantly influenced by tropical
climate with the heaviest rainfall happening between October and December. November is the
peak of this season where 324 mm of rainfall is experienced across 28 days. In October 222
mm of rainfall is experienced while in December 246 mm of rainfall is experienced. At the
beginning of the year the amount of rainfall is relatively lower. January and February receive
148 mm and 102 mm respectively. However, April receives a rainfall of 241 mm which is
comparable to precipitation received in peak season. During the summer months of June and
July relatively lower rainfall amounts of 145 mm and 135 mm respectively are received
\cite{nomadseason_2025}. These seasonal patterns have a major influence on local ecosystems as
well as agriculture activities and water management. The Malaysian Meteorological
Department \cite{malaysian_meteorological_department_2025} analysed annual rainfall data from 1951 to 2023 and found there has been
an upward trend in the amount of rainfall received in the country. This points to climate change
that can lead to higher temperatures, rising sea levels, often occurrence of extreme events such
as floods, disruption of habitats and agricultural activities, and economic losses. These
fluctuations make it difficult to accurately forecast climate patterns. Climatic events such as
frequent and heavy rainfall can lead to crop failure, floods, and water contamination. Similarly,
seasons such as the monsoon have a significant influence on rainfall and its distribution. The
Department of Irrigation and Drainage Malaysia \cite{department_of_irrigation_and_drainage_malaysia_2025} describes monsoon rains as “typically
of long duration with intermittent heavy bursts and the intensity can occasionally exceed
several hundred mm in 24 hours”. This can lead to floods in urban areas and disruption of
agricultural activities. Accurate forecasting will help the Selangor State Government in
mitigating the effects of these events. Equipped with accurate forecasts the state government
can put in place well planned emergency as well as disaster and preparedness strategies.

Machine learning models have become a critical tool in analysis of meteorological data. When
comparing machine learning models with conventional Numerical Weather Prediction models,
it has been observed machine learning models are superior at detecting intricate numerical and
non-linear patterns in data \cite{bouallegue_et_al_2024}. This makes machine learning a suitable
approach for predicting rainfall in a tropical region like Selangor. Large amounts of
meteorological data can be analysed using machine learning techniques such as support vector
machines (SVM), gradient boosting, and artificial neural networks (ANN) to provide accurate
temporal estimations. These methods that will be discussed later, use historical data such as
temperature, humidity, wind speed, and rainfall to provide accurate forecasts which were
hitherto impossible using traditional techniques such as linear regression.

The problem is critical in places such as Selangor, where rainy conditions have not been
accurately forecasting posing several difficulties. Hydrological functions enhanced by better
rainfall predictions enable timely decisions in crop production, disaster management including
floods and landslides, and water management. Due to improved accuracy levels of predictions,
stakeholders will be in a position to save structures from destruction, people from hunger as
well as resources from wastage.

Recent advances in machine learning have expanded possibilities for improving rainfall
forecasting. Machine learning methods like support vector machines, gradient boosting, and
artificial neural networks have shown great potential in capturing both temporal and spatial
patterns of rainfall. These models are able to improve forecasts by continuously learning from
new data. In Selangor, using machine learning techniques and local meteorological data
presents an opportunity to develop a forecasting system that is highly accurate.

\subsection{Problem Statement}
Climate change has received significant global attention due to disastrous events it can
cause. Rainfall is a major meteorological factor that is influenced by climate change. In
Malaysia, rainfall patterns have changed causing floods and droughts. Selangor is one the states
that has been affected by these changes in rainfall patterns. Disastrous floods happened
consecutively in the years 2006 to 2008 and in the years 2010 and 2011. The years 1997, 1998,
and 2008 had catastrophic dry periods \cite{talib_et_al_2024}. Agricultural decisions and
productivity are significantly influenced by environmental variables particularly the amount of
water available and rainfall. In Selangor the influence of these variables is significant and a
threat to agricultural productivity. High and low rainfall affects crops. Although it is possible
to mitigate low rainfall through irrigation, high rainfall usually damages crops and results in
low agricultural productivity. Mitigation measures such as changing crop cycles and combining
crop cycles have not been adequate. To adequately solve these problems technological
solutions are required \cite{alam_2021}.

One of the technological solutions that can be used is availing accurate rainfall
predictions. However, due to irregular occurrence of rainfall in Timur Region Selangor
accurate prediction is difficult. This situation can harm farming, cause floods, and cause
difficulties in water resources planning. Traditional models such as linear regression may not
provide accurate precipitation forecast especially in the tropics because the atmospheric
behaviour is not easy to predict. For example, Kassem et al. \cite{kassem_et_al_2021} reported artificial neural
networks were superior to linear regression in predicting monthly rainfall in Northern Cyprus.
That study showed artificial neural networks were better at capturing relationships in
coordinates, meteorological variables, and rainfall resulting in more accurate prediction
compared to linear regression. Traditional models such as linear regression are weak at
capturing complex relationships especially when they are non-linear. Compared to models such
as support vector machines and artificial neural networks, linear regression models are poor at
handling non-linear relationships. Conversely, support vector machines and artificial neural
networks are difficult to interpret, computationally costly, and require large amounts of data
\cite{goodfellow_et_al_2023; murphy_2022}. Modern meteorological research does not face the
limitations of small datasets and limited computational power that were prevalent several
decades ago. Meteorological instruments and IoT sensors have enabled accumulation of large
datasets. This situation enables use of advanced machine learning models such as support
vector machines and artificial neural networks in predicting rainfall. Specifically, in Selangor
large volumes of meteorological data are available. Therefore, these advanced machine
learning models can be used to accurate predict rainfall patterns. Insights obtained will be
useful in agricultural, infrastructure, public health, and water management planning.

\subsection{Research Questions}
The specific research questions that will be investigated in this study are:
\begin{enumerate}
    \item What are the machine learning models that can be used for rainfall prediction in
Selangor?
    \item How does the performance of different machine learning models differ?
    \item What is the best model in forecasting rainfall pattern in Selangor?
\end{enumerate}

\subsection{Research Objectives}
The broad objective of this study is to investigate the use of machine learning models in
predicting rainfall in Selangor region of Malaysia. The specific objectives are:
\begin{enumerate}
    \item To employ machine learning models that can be used for predicting rainfall in
Selangor.
    \item To estimate and assess the performance of different machine learning models using
performance metrics such as mean absolute error (MAE), mean squared error (MSE),
root mean squared error (RMSE), and R-squared.
    \item To identify the best model for forecasting rainfall patterns in Selangor by comparing
performance metrics and selecting the model with highest accuracy.
\end{enumerate}

\subsection{Research Scopes}
This research deals with rainfall prediction for Selangor, Malaysia where the rainfall has
irregular tropical pattern and significantly affects sectors such as water supply and flood
control, agriculture. These problems will be addressed in this work by utilising and comparing
a number of machine learning algorithms with support vector machines (SVM), gradient
boosting, and artificial neural networks (ANN). These methods were chosen due to the
possibility of the interpretation of which dependencies – both linear and nonlinear ones – are
present in the data. In the present study, meteorological data from Sepang/KL International
Airport is employed for data analysis where necessary climatic factors embracing average
temperature, relative humidity, wind velocity, and precipitation for the years between 2012 and
2020 are utilised. This is to make certain that the data collected are accurate and reliable to
increase the efficiency of data analysis after it has been fed into the system therefore data
cleaning, normalization of data, handling of missing values and feature engineering will be
undertaken.To fully assess predictive performance, the model will be evaluated using measures
like the Coefficient of Determination ($R^2$), Mean Absolute Error (MAE), and Root Mean
Square Error (RMSE).

\subsection{Significance of Study}
The research focus on using machine learning for rainfall forecasting in Selangor.
Machine learning techniques utilize historical data to identify complex relationships, resulting
in more precise and current forecasts. This study improves the scientific understanding of based
on rainfall forecasting by evaluating how well different machine learning algorithms capture
detailed tropical rainfall patterns. It represents a major breakthrough in environmental
prediction and building resilience since it expands the use of machine learning for tropical
weather forecasting and offers a structure that can be adjusted for different climates. The
forecasting results could help the government in enhancing disaster readiness.

\section{Literature Review}

\subsection{Introduction}

In tropical regions such as Selangor in Malaysia where extreme events such as high or
low rainfall happen; accurate rainfall forecasting is critical. When managers are provided with
accurate predictions, they are better placed to put mitigation measures in place. These measures
can help in management of disruptive events such as floods, agricultural crop failure, and
disruptions in water supply. Machine learning has emerged as a powerful technique for
analysing rainfall data, discovering patterns in meteorological data, and accurately predicting
rainfall. This chapter presents an exhaustive review of existing literature on use of machine
learning for forecasting rainfall. Specifically, the strengths and limitations of each study are
evaluated to identify research gaps that can be addressed in this study and future studies.

\subsection{Challenges in Rainfall Forecasting}

Numerous studies have well documented challenges faced when predicting rainfall.
Kundu et al. \cite{kundu_et_al_2023} have discussed some of these challenges. These authors note the primary
challenge is the wide variability in rainfall patterns. Other challenges are scarcity of relevant
meteorological variables such as soil, humidity, wind and temperature parameters which are
essential. When these variables are not available the accuracy of prediction models is severely
affected. Other human activities such as deforestation can also negatively affect the accuracy
of rainfall prediction models. Even when advanced methodologies are used accurate prediction
of rainfall is challenging as large volumes of data and collaboration are required.

The National Oceanic and Atmospheric Administration \cite{national_oceanic_and_atmospheric_administration_2024} notes forecasting
weather phenomena is a difficult skill that requires meticulous observation and analyzing large
amounts of data. Weather phenomena can be characteristically thunderstorms covering large
areas or a small area that can last for a few hours or several days. The phases involved in
weather forecasting are observation, prediction, and dissemination of results.

Ray et al. \cite{ray_et_al_2021} discuss various challenges faced in predicting rainfall driven by
landfalling tropical cyclones in India. Rainfall from these tropical cyclones especially when
approaching landfall varies widely and is usually asymmetric. This pattern is often caused by
wind, speed, land surface, and moisture parameters. That study found that increase or decrease
in intensity of a tropical storm as it approached the coastline during landfall can change the
characteristics of rainfall over land.

Selangor is a typical tropical environment characterized by widely fluctuating rainfall
patterns. This variation makes accurate rainfall prediction challenging. These challenges arise
because rainfall patterns are influenced by intricate relationships among atmospheric factors
like variations temperature, humidity, and windspeed. Rainfall predictions are usually obtained
from large scale computerized simulations of weather systems. Use of traditional prediction
methods like numerical weather prediction fails at capturing events that happen in isolated
areas. Furthermore, this problem is severe in areas that have widely varying rainfall patterns
such as Selangor. These models are further limited by their high cost and their lack of flexibility
to adjust to changes in rainfall patterns in real time. Machine learning is a viable alternative for
overcoming challenges faced by these traditional models. Particularly, machine learning
models are suited to capturing complex and non-linear relationships that exist in meteorological
data. With these capabilities machine learning models are an essential tool for discovering
patterns that exist in historical meteorological data.

\subsection{Overview of Machine Learning Techniques for Rainfall Prediction}

Machine learning models are well suited to capture non-linear relationships that are a
common feature in meteorological variables like temperature, windspeed, humidity, and
precipitation. This makes machine learning models a robust technique for analysing
meteorological data. This section presents an exhaustive review of literature that has examined
use of different machine learning models for rainfall prediction.

\begin{table}[h]
\centering
\caption{Table of Summary}
\begin{tabular}{llll}
\toprule
Authors & Techniques & Data Frequency & Main Result \\
\midrule
Praveena et al. (2023) & Support vector machines, Logistic Regression & Daily & Both techniques achieve optimized results after hyperparameter tuning. \\
Hayaty et al. (2023) & Support vector machines & Daily & Support vector machine had an accuracy of 72\% \\
Hapsari et al. (2020) & Support vector machines & Daily & Stochastic gradient optimization had better performance compared to time series \\
Yin et al. (2022) & QM, CDFt, support vector machines & Monthly & A hybrid SVM-QM model outperformed the other models \\
Al-Mahdawi et al. (2023) & Support vector machines & Monthly & Support vector machines had low MAE, RMSE, and MSE in some months but useful forecasts were obtained \\
Du et al. (2021) & Support vector machines & Daily & Swarm optimization was useful for improving accuracy \\
Velasco et al. (2022) & Support vector machines & Monthly & A radial basis kernel produced acceptable accuracy as measured by MSE \\
Nuthalapati. (2024) & Decision tree, K-Nearest Neighbor, Random Forest, Gradient Boosting, Logistic Regression & Daily & Gradient Boosting and Logistic Regression achieve the highest accuracy of 80.95\% \\
Anwar et al. (2020) & XGBOOST & Daily & Best RMSE was obtained at five iterations \\
Poola and Sekhar (2021) & XGBOOST & Monthly & Model had a high accuracy of 95\% \\
Nuthalapati and Nuthalapati (2024) & KNN, SVM, gradient boosting, XGBOOST, logistic regression, random forest & Daily & XGBOOST had superior performance compared to the other models \\
Cui et al. (2021) & SSA, LightGBM & Daily & A hybrid SSA-LightGBM was superior to either model \\
Sanches et al. (2024) & XGBOOST & Daily & An accuracy of 90\% in classification and MAE of 3mm in regression were observed \\
Maaloul and Leidel (2023) & Random forest, decision tree, naïve bayes, gradient boosting, neural networks & Daily & Gradient boosting had the highest accuracy \\
Zhuang and DeGaetano (2024) & LightGBM & Daily & LightGBM had similar performance to random forest and gradient boosting but had higher accuracy than KNN and linear kernel SVM \\
Raniprima et al. (2024) & Random forest, decision tree & Daily & Random forest had a marginally higher accuracy than decision tree \\
Hsu et al. (2024) & Random forest, CatBoost & Daily & Random forest had better performance compared to CatBoost \\
Raut et al. (2023) & random forest regression, linear regression, support vector regression, and decision trees & Daily & Random forest had best performance compared to the other models \\
Sanaboina. (2024) & Artificial Neural Network & Daily & Yield accuracy of 88.65\% \\
Primajaya and Sari (2021) & Random forest & Daily & MAE and RMSE values of 0.35 and 0.46 were observed \\
Bhardwaj and Duhoon (2021) & “Quinlan M5 algorithm, reduced error pruning tree, random forest, logit boosting, Ada boosting” & Monthly & Random forest had best performance \\
Resti et al. (2023) & Decision tree & Daily & An accuracy of 98.53\% was observed \\
Sharma et al. (2021) & Decision tree & Daily & Decision trees are useful for risk evaluation \\
Nurkholis et al. (2022) & C5.0 decision tree & Daily & A high accuracy was observed \\
Kaya et al. (2023) & Feed forward neural network & Daily & An accuracy of 93.55\% and RMSE of 0.254 were observed \\
Mislan et al. (2022) & Back propagation neural network & Monthly & MSE of 0.00096 was observed \\
Aizansi et al. (2024) & Multi-layer perceptron neural network, LSTM, climatology forecasts & Monthly & Multi-layer perceptron outperformed LSTM \\
Ejike et al. (2021) & Logistic regression & Daily & An accuracy of 84\% was observed \\
Khan et al. (2024) & “Logistic regression, decision trees, multi-layer perceptron, and random forest” & Daily & Logistic regression had highest accuracy \\
Moorthy and Parmershawaran (2022) & WOAK, KNN & Daily & Hybrid model consisting of WOAK and KNN outperformed either model \\
Huang et al. (2020) & WKNN, support vector machine & Daily & WKNN was at par with support vector machine \\
Lee et al. (2022) & Artificial neural network & Monthly & RMSE value of 34.75\% was observed on test subset \\
Findawati et al. (2021) & “Naïve Bayes, K-nearest neighbor, and C4.5” & Daily & KNN had highest accuracy \\
Yu and Haskins (2021) & “Deep neural network, wide neural network, deep and wide neural network, reservoir computing, long short term memory, support vector machine, and K-nearest neighbor” & Monthly & KNN had highest MSE and RMSE \\
Setya et al. (2023) & Linear regression, KNN & Monthly & KNN had better RMSE and MAE compared to linear regression \\
Dawoodi and Patil (2020) & KNN & Daily & An accuracy of 96\% was observed \\
Wolfensberger et al. (2021) & Random forest, QPE & Daily & Random forest was better than QPE \\
\bottomrule
\end{tabular}
\end{table}

\section{Methodology}

\subsection{Introduction}

This chapter presents the steps that will be followed in identifying the machine learning
algorithm that provides the highest accuracy in predicting rainfall in Selangor. The steps
involved are exhaustive review of available literature, identifying the problem to be
investigated, collecting relevant data, pre-processing data to assure its suitability, model
training, tuning model parameters, and evaluating models. This structured approach will ensure
all critical steps are followed. It is expected this approach will help in meeting study objectives.

\subsection{Research Design}

This research design will act like a blueprint that will be followed in every stage of the
study. The core objective is to compare machine learning algorithms and identify the algorithm
that provides the highest prediction accuracy. A data driven approach is followed whereby
historical weather data such as precipitation, temperature, humidity, and windspeed are the
foundation of the study. A data science lifecycle that involves data gathering, pre-processing,
parameter tuning, and model evaluation is followed.

\subsection{Data Science Methodology}

%\begin{figure}[h]
%\centering
%\includegraphics[width=0.8\textwidth]{../figures/data_science_methodology.png} % Assuming this figure exists or will be created
%\caption{Data Science Methodology}
%\label{fig:data_science_methodology}
%\end{figure}

\subsubsection{Literature Review}

The first step in carrying out a study is reviewing available literature. Extant literature on
machine learning models used for predicting rainfall was reviewed. From reviewed literature
it was evident machine learning is an established technique in rainfall forecasting. Reviewed
literature revealed machine learning models are primarily used for forecasting the amount of
rainfall or classifying rainfall to several categories such as rain/no rain or intensity of rainfall
such as low/medium/high. To a lesser extent machine learning were also used to identify
critical factors that affect rainfall. Commonly used machine learning methods were support
vector machines, decision trees, K-nearest neigbour, logistic regression, gradient boosting,
XGBOOST, linear regression, and artificial neural networks. With the exception of logistic
regression all the other machine learning models can be used to predict a quantitative amount
of rainfall. It was evident in almost all studies a train and test subset were used. This provides
a subset for training the model and another subset not used for training that will be used to
evaluate model performance. Reviewed literature showed data preprocessing steps such as
checking missing values, imputing missing values, checking out of range values, and
normalizing quantitative variables to a common range are critical to performance of a machine
learning model. From the literature it was observed that some machine learning models have
hyperparameters that need to be tuned to achieve high prediction accuracies. These principles
that are well established in the literature will be incorporated in this study.

\subsubsection{Problem Identification}

Climate change has resulted in disruption of established weather patterns. This is a global
phenomenon that can lead to extreme rainfall events such as too little or too much rainfall.
These events have significant impact on public health, infrastructure, and agriculture. Although
economic activities in Selangor are not primarily agricultural, extreme rainfall events need
proper planning and mitigation. As a largely urbanized area, flooding from extreme rainfall
events such as too much rainfall can cause major disruptions in infrastructure such as public
transport, water supply, and waste management. Similarly, too little rainfall can disrupt water
supply in urban areas. In rural areas of Selangor where crops such as palm and rubber are grown
as well as livestock rearing, these extreme rainfall events can be debilitating. Too little or too
much rainfall can cause crop failure. Literature reviewed showed mitigation measures such as
changing types of crops or crop cycles were not adequate. These challenges make accurate
rainfall prediction an essential strategy in planning and management within the Selangor state
government. It is these challenges that were the main motivation of this study. This study aims
to investigate if machine learning models can be used to produce accurate rain forecasts. These
forecasts will be extremely useful to state government planners.

\subsubsection{Data Collection}

A dataset consisting of five variables which are date, average temperature, wind speed, relative
humidity and precipitation will be used. Use of these variables is well established in the
literature. The target variable will be precipitation and the main objective of this study is to
evaluate performance of machine learning models in predicting this variable. The predictors
will be the other variables except date. The date variable will be useful in building time series
models such as ARIMA. The selected dataset consists of daily observations covering the period
between 2012 and 2020.

\subsubsection{Data Preprocessing}

The selected dataset is expected to have some data quality issues. Exploratory data analysis
will be used to identify missing values, values that are not within the expected range, and to
understand the distribution of variables. Any missing values will be replaced with the mean
value to avoid altering the distribution of variables. Any values that are not withing the
expected range will be dropped in the analysis. To ensure all variables have an equal
contribution to the model, each variable will be normalized. This will ensure all variables have
a common range. In addition, the original daily data were combined into weekly data to reduce
noise and show bigger trends in climate behaviour. A ratio of 80\% to 20\% will be used to split
the dataset into train and test subsets. The train subset will be used for model training while the
test subset will be used for model evaluation. These principles are well established in reviewed
literature.

\subsubsection{Model Training}

The models that will be investigated in this study are: artificial neural networks, support vector
machines, decision trees, multiple linear regression, K-nearest neighbour, random forests,
gradient boosting, and ARIMA. With the exception of linear regression all the other models
have a set of parameters that will need to be tuned to achieve the highest prediction accuracy.
These parameters are discussed for each model.

The artificial neural network has three architectural parameters that specify the general
structure. They are layers, neurons in each layer, and activation functions. The layers and
number of neurons will be used to achieve a balance between overfitting and long training time.
Activation functions such as ReLu, Tanh, and Sigmoid will be used to capture non-linear
patterns in the data. Various training parameters such as learning rate, batch size, epochs, and
optimization methods such as SGD, RMSprop, and Adam will be examined to understand their
influence on model accuracy. The dropout rate, L1, and L2 will be used to control overfitting.

The hyperparameters of a support vector machine that will need tuning are kernel,
regularization, and epsilon. A non-linear relationship is expected in the data. Therefore, only
radial basis and polynomial kernels will be examined. The regularization parameter will be
tuned to control overfitting in the model. Epsilon will be tuned to control prediction accuracy.

The K-nearest neighbour hyperparameters that will be tuned are neighbours and distance
metrics. The number of neighbours will be used to control overfitting. Various distance metrics
such as Euclidean, Manhattan, and Minkowski will be examined.

The random forest hyperparameters that will be tuned are: maximum depth, samples per
leaf/tree, maximum features/leaf nodes, and split criterion. Tuning will ensure the model
adequately captures the relationships in the data while avoiding overfitting or underfitting.

Gradient boosting parameters such as trees, learning rate, depth, split, subsampling, and
features will be tuned to minimize overfitting and maximize prediction accuracy.

An ARIMA model requires optimal identification of p, d, and q parameters. Visual inspection
and stationarity tests will be used to identify an optimal differencing order. The autocorrelation
and partial autocorrelation functions will be used to identify optimal p and q parameters.

The R statistical software will be used for exploratory data analysis and model training. This
software was selected because it is freely available and provides extensive data visualization
and algorithm capabilities.

\subsubsection{Model Evaluation and Comparison}

Three model evaluation metrics which are Root Mean Squared Error (RMSE), Mean Absolute
Error (MAE), and the Coefficient of Determination ($R^2$) will be used to examine performance
of models under investigation.

RMSE captures the square root of the average squared differences between predicted and actual
observations. It shows the extent of large errors and is useful for identifying large deviations
in rainfall predictions. RMSE is easy to interpret as it is expressed in units of the response
variable but has the limitation of not adequately capturing the influence of outliers. The formula
for RMSE is:

\begin{equation*}
\text{RMSE} = \sqrt{\frac{1}{n}\sum_{i=1}^{n}(y_i - \hat{y}_i)^2}
\end{equation*}

Where:

\begin{itemize}
    \item $y_i$: Actual of observation i
    \item $\hat{y}_i$: Prediction of observation i
    \item $n$: Number of observations
    \item $\Sigma$: Summation from 1 to i
\end{itemize}

MAE captures the average difference in the absolute predicted and actual values. This provides
a simple measure of prediction accuracy. MAE differs from RMSE as it considers all errors
equal, making it robust against outliers. The formula for MAE is:

\begin{equation*}
\text{MAE} = \frac{1}{n}\sum_{i=1}^{n}|y_i - \hat{y}_i|
\end{equation*}

Where:

\begin{itemize}
    \item $y_i$: Actual of observation i
    \item $\hat{y}_i$: Prediction of observation i
    \item $n$: Number of observations
    \item $\Sigma$: Summation from 1 to i
\end{itemize}

$R^2$ captures the extent to which the model explains the variation in the target variable. An $R^2$
value close to 1 shows the model is very good at capturing a high degree of the variation, while
a value close to zero is indicative of poor predictive performance. The formula for $R^2$ is:

\begin{equation*}
R^2 = 1 - \frac{\sum_{i=1}^{n}(y_i - \hat{y}_i)^2}{\sum_{i=1}^{n}(y_i - \bar{y})^2}
\end{equation*}

Where:

\begin{itemize}
    \item $y_i$: Actual of observation i
    \item $\hat{y}_i$: Prediction of observation i
    \item $n$: Number of observations
    \item $\Sigma$: Summation from 1 to i
\end{itemize}

These metrics will be very helpful in understanding the models under investigation. The MAE
and RMSE will provide a quantitative value that indicates the difference between actual and
predicted rainfall values. This will be useful in identifying the model that provides the best
accuracy. $R^2$ indicates the extent of model overfitting or underfitting. Therefore, comparison of
these three metrics will provide a comprehensive performance evaluation.

Tables will be used to present the performance metrics of each model. This will facilitate easy
comparison of the various models.

\subsubsection{Deployment}

The selected machine learning model will be deployed as a prototype application to
demonstrate its practical use. This application could be integrated into an early warning system
or a web-based platform to provide real-time rainfall forecasts for stakeholders such as farmers,
city planners, and disaster management authorities. Deployment may involve creating a
Python-based application with APIs to deliver actionable insights effectively.

\section{Expected Outcomes and Conclusions}

\subsection{Introduction}

This chapter will present the expected outcomes from the study. After carefully following
the methodology developed earlier all study objectives will be achieved. The broad objective
of the study is to investigate the potential of using machine learning in planning and
management of extreme rainfall events in Selangor. Insights obtained from machine learning
predictions will be used for agriculture, disaster, and water management planning.

\subsection{Expected Outcomes}

This study is expected to meet its objectives. The first objective is to employ machine
learning for rainfall prediction. This objective has been addressed through a comprehensive
review of existing literature, which demonstrates the effectiveness of machine learning
algorithms such as artificial neural networks, support vector machines, random forests, linear
regression, K-nearest neighbours, gradient boosting, and ARIMA in forecasting rainfall. The
literature also emphasizes the importance of practices such as data quality checks, data
normalization, and appropriate train/test splits for ensuring model accuracy. Additionally,
widely used evaluation metrics including RMSE, MAE, and R-squared will be adopted in this
project to assess model performance.

The second objective will be to train identified machine learning algorithms using the
data specified in the methodology. This objective has not been achieved. The methodology
specified earlier will be followed in training each of the selected models. It is expected careful
tuning of parameters will train models that balance computational cost, accuracy, and
overfitting.

The third objective will be to identify the machine algorithm that provides the highest
accuracy in rainfall prediction. This objective has not yet been met and it will only be achieved
after examining results from objective 2. After training the models on the train subset, the
performance of the models on the test subset will be examined using evaluation metrics and
test subset. It is expected comparison of evaluation metrics will identify the algorithm with the
highest accuracy.

%\begin{figure}[h]
%\centering
%\includegraphics[width=0.8\textwidth]{../figures/random_forest_regression.png} % Assuming this figure exists or will be created
%\caption{Random Forest Regression: The blue dots represent the actual total precipitation per year, while the orange line shows the predicted values from the model. The model uses yearly averages of temperature, humidity, and wind speed to estimate total precipitation.}
%\label{fig:random_forest_regression}
%\end{figure}

\begin{figure}[h]
\centering
\includegraphics[width=0.8\textwidth]{../figures/correlation_matrix.png} % Assuming this figure exists or will be created
\caption{Correlation Matrix: The correlation matrix shows some important relationships between the weather variables. Temperature and humidity have a strong negative relationship, meaning when the temperature goes up, humidity usually goes down. Temperature also has a moderate positive link with wind speed, so higher temperatures often come with stronger winds. There is a weak negative connection between temperature and rainfall, suggesting that hotter days tend to have less rain. Rainfall and the “Rain Today” variable have a moderate positive link, which makes sense since more rain usually means it rained that day. The week and year don’t strongly affect the other variables, but they may still help track changes over time. Overall, temperature, humidity, and wind are useful for predicting rainfall.}
\label{fig:correlation_matrix}
\end{figure}

\subsection{Conclusions}

In conclusion, this research will build and evaluate machine learning models capable of
accurately forecasting rainfall in Selangor. Using weather data and appropriate machine
learning algorithms it is expected this study will identify a machine learning algorithm that can
be incorporated into an early warning system. Such an early warning system will be critical to
success of agriculture, infrastructure, and water management planning within Selangor. This
study will demonstrate the value and limitations of using machine learning algorithms in
rainfall prediction.

The findings are expected to provide actionable insights for various stakeholders, enabling
better resource management, flood prevention, and agricultural planning. However, just like
any other study this study will also have limitations. These limitations will only be fully clear
after the project is completed. The findings of this study will then require interpretation in
consideration of limitations.

\section*{Acknowledgements}
This research was supported by the Selangor State Government and Universiti Malaysia Pahang Al-Sultan Abdullah.

\bibliographystyle{plain}
\bibliography{references}
\end{document}

\documentclass{article}
\usepackage{pgfplots}
\pgfplotsset{compat=1.18}
\usepackage{booktabs}
\usepackage{siunitx}
\usepackage{graphicx}

\title{Rainfall Forecasting in Selangor Using Machine Learning Techniques}
\author{Data Science Team}
\date{\today}

\begin{document}
\maketitle

\section{Introduction}
This report presents the analysis and forecasting of rainfall patterns in Selangor, Malaysia using various machine learning techniques. 

\section{Methodology}
We employed the following models for rainfall prediction:
\begin{itemize}
    \item Multiple Linear Regression
    \item K-Nearest Neighbors (KNN)
    \item Random Forest
    \item XGBoost
    \item Artificial Neural Network (ANN)
    \item ARIMA
\end{itemize}

\section{Data}
The analysis used weekly rainfall data from 2012-2021, including temperature, humidity, and wind speed measurements.

\section{Results}
Performance metrics for each model will be displayed in the following sections.

\subsection{Model Comparison}
\begin{table}[h]
\centering
\caption{Model Performance Metrics}
\label{tab:model_performance}
\begin{tabular}{l c c c c}
\toprule
Model & MAE & MSE & RMSE & R\textsuperscript{2} \\
\midrule
% RESULTS_TABLE %
\bottomrule
\end{tabular}
\end{table}

\subsection{Visualizations}
\begin{figure}[h]
\centering
% ACTUAL_VS_PREDICTED_PLOT %
\caption{Actual vs Predicted Rainfall}
\end{figure}

\begin{figure}[h]
\centering
% ERROR_DISTRIBUTION_PLOT %
\caption{Error Distribution}
\end{figure}

\section{Conclusion}
The best performing model was % BEST_MODEL % with an R\textsuperscript{2} of % BEST_R2 %.
\end{document}
